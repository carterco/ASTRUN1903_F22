\documentclass[10pt]{article}% uses letterpaper by default

%---------- Uncomment one of them ------------------------------
\usepackage[includeheadfoot, top=1in, bottom=1in, hmargin=1in]{geometry}

% \usepackage[a5paper, landscape, twocolumn, twoside,
%    left=2cm, hmarginratio=2:1, includemp, marginparwidth=43pt, 
%    bottom=1cm, foot=.7cm, includefoot, textheight=11cm, heightrounded,
%    columnsep=1cm, dvips,  verbose]{geometry}
%---------------------------------------------------------------
\usepackage{fancyhdr}
\usepackage{verbatim}
\usepackage{url}
\pagestyle{fancy}
\usepackage{graphicx}
\usepackage{setspace}
%\doublespacing
\singlespacing
%\onehalfspacing
%\newcommand{\exercisename}{}

\lhead{Astronomy Lab}
\chead{}
\rhead{AMNH Visit}
\renewcommand{\rightmark}{}
\lfoot{Joo Heon Yoon and Courtney Carter} \cfoot{\thepage} \rfoot{Fall 2022}

\newcommand{\degrees}{\ensuremath{^\circ}}
\newcommand{\arcmin}{\ensuremath{'}}
\newcommand{\arcsec}{\ensuremath{"}}
\newcommand{\hours}{\ensuremath{^\mathrm{h}}}
\newcommand{\minutes}{\ensuremath{^\mathrm{m}}}
\newcommand{\seconds}{\ensuremath{^\mathrm{s}}}

\begin{document}
\begin{center}
\huge Meteorites and the Early Solar System
\end{center}

\vspace{0.3cm}

\begin{flushleft}

There's a lot more to the Solar System than just planets! In the Hall of
Meteorites at the Museum of Natural History, we can learn about the other solid
bodies that orbit the Sun. Although smaller, they have played a role in the
evolution of life on Earth, and they carry records that tell the history of the
Solar System going all the way back to the formation of the planets and even
further.  

\vspace{0.3cm}

\textbf{\emph{After} taking time to explore all of the exhibits in the 
Hall of Meteorites, 
\emph{then} answer the questions below in your lab book. Some answers require 
you to combine information from more than one exhibit.} 

\vspace{0.3cm}

\begin{center}
\textbf{I. The Ahnighito Meteorite}
\end{center}

\vspace{0.3cm}

\begin{enumerate}
\item What makes this meteorite so special for it to be the centerpiece of the 
collection here at AMNH?

\vspace{0.3cm}

\item Diagram the relative size of the Ahnighito meteorite by making a rough
sketch of its profile along with a person standing next to it. 

\vspace{0.3cm}

\item What is the meteorite mostly made of?  How heavy is the Ahnighito meteorite?
If you assume that an average person weighs 150 lbs, and a ton is 2000 lbs, then
how many people does it take to be as heavy as this relatively compact 
meteorite?

\vspace{0.3cm}

\item What is the difference between a meteorite and a meteor? 

\vspace{0.3cm}

\item Give an example of a good place on the Earth to search for meteorites.

\vspace{0.3cm}

\begin{center}
\textbf{II. Impacts}
\end{center}

\vspace{0.3cm}

\vspace{0.3cm}

\item Why is the Moon's face covered in craters, while we only see a few on Earth? 
Explain.

\vspace{0.3cm}

\item How does the impact of a large body lead to a mass extinction, as in the
case of the dinosaurs?  

\vspace{0.3cm}

\item When and where do scientists think Chicxulub (the dinosaur-killing impact) 
occurred?  What is the evidence for it being responsible for this mass
extinction?

\vspace{0.3cm}

\item Describe one plausible way to divert the path of an asteroid on a collision
course with Earth that does not involve Bruce Willis.

\vspace{0.3cm}

\item Briefly explain the leading theory of how the Moon formed. 

\vspace{0.3cm}

%\newpage

\begin{center}
\textbf{III. The Early Solar System}
\end{center}

\vspace{0.3cm}

\item What is the solar nebula? 

\vspace{0.3cm}

\item If you were to take a small amount of the solar surface and cool it to
room temperature, it would have a composition similar to what objects?  Why?

\vspace{0.3cm}

\item All in all, what do \emph{you} consider to be the main reasons scientists
would want to study meteors/meteorites?  Please cite two reasons.

\vspace{0.3cm}

\begin{center}
\textbf{IV. Worlds Beyond Earth (Post space-show questions)}
\end{center}

\vspace{0.3cm}

\item What did you learn about the early solar system from space show? 

\vspace{0.3cm}

\item Describe something about the evolution of the solar system that you didn't know before.

\vspace{0.3cm}

\begin{center}
\textbf{V. Reflection}
\end{center}

\vspace{0.3cm}

\item Write a brief (2-4 sentences) reflection on what you took away from this lab. If you need some inspiration, what was your favorite part of visiting The Rose Center for Earth and Space and why? Alternatively/in addition, describe something new you learned.

\end{enumerate}
\end{flushleft}
\end{document}