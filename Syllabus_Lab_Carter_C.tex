\documentclass[11pt]{article}
\usepackage[includeheadfoot, top=1.0in, bottom=1.0in, hmargin=1.0in]{geometry}
\usepackage[utf8]{inputenc}
\usepackage{fancyhdr}
\usepackage{url}
\pagestyle{fancy}
\usepackage{setspace}
\usepackage{tabularx}

\lhead{Astronomy UN1403}
\rhead{Fall 2022}

\title{Astronomy Lab I (ASTR UN1403)\\ \medskip \Large{Fall 2022 Syllabus}}
%\huge {Fall 2021 Syllabus}
\author{}
\date{}

\begin{document}
\maketitle
%\begin{center}
%{\huge ASTRONOMY LAB II}\\
%\medskip
%{\Large Wed 7 - 10pm}\\
%{\Large Pupin 1424}
%\end{center}
\begin{tabularx}{0.9\textwidth} { 
   >{\raggedright\arraybackslash}X 
   >{\raggedright\arraybackslash}X 
   >{\raggedright\arraybackslash}X  }
 \textbf{Instructor:} & \textbf{Office:} & \textbf{Email:} \\ 
 \hline
Courtney Carter & Pupin 1424 & \url{cjc2298@columbia.edu} \\  
%  Jennifer Mead & Pupin 1333 & \url{jm4793@columbia.edu}  \\
\end{tabularx} 

\bigskip
\bigskip
%\noindent \textbf{\normalsize Instructors: \hspace{3.cm} Office:  \hspace{3.5cm} Email:} \\
%\normalsize Shifra Mandel \hspace{3.cm} Pupin 1333 \hspace{3.cm}  \url{ss5018@columbia.edu} \\ 
%{\normalsize Jennifer Mead \hspace{3.cm} Pupin 1411 \hspace{3.cm}  \url{jm4793@columbia.edu}} \\ 

 
 
\noindent \textbf{\normalsize Time:} {\normalsize Wednesday 7:00-10:00 p.m.} \\
\noindent \textbf{\normalsize Location:} {\normalsize Astronomy Library, Pupin 1402} \\
\textbf{\normalsize Office hours:} {\normalsize By appointment} \\
 
 
\section*{Class Overview}
 
Welcome to the astronomy lab! The objectives of this lab are for you to:
\begin{itemize}
\item Demystify the scientific method, develop critical thinking skills, and learn to apply scientific reasoning in your evaluation of information and arguments. 
\item Develop a better sense of the scale of the Universe, an understanding of error and uncertainty in measurements, and other quantitative tools. %that will serve you beyond the scope of this lab. 
\item Experience the scientific process of framing and asking a well-defined question, gathering data and testing that question quantitatively, and communicating your work -- textually and verbally -- to your peers.
\end{itemize}

\bigskip
 
\noindent There will be 10 labs between September 14th and December 7th; no lab on November 23 (Thanksgiving break). There will be no work assigned outside of these sessions. Our final lab session will be devoted to 10-minute student presentations.
%For our final lab session, everyone is required to give a 10-minute presentation (followed by discussions for 5 min). A list of topics related to astronomy and science in society will be rolled out; you are also welcome to submit your own suggested topics for instructor approval. The presentation will be worth the weight of two lab sessions. 

\noindent Don't shy away from discussions with your peers and instructors. Don't hesitate to ask questions and don't be afraid to make mistakes; done appropriately, those are very effective ways to learn. And importantly, don't forget to have fun!
 
\section*{Lab Materials}
 
Please bring the following to each lab session (including the first):
 
\begin{itemize}
\item \textbf{A lab notebook:} This can be a bound notebook, but feel free to use a digital version instead; if you choose the latter, make sure to have your device with you and ready to use.  
\item \textbf{Writing/drawing tools:} Pen, pencil, eraser, ruler, etc. Colored pens/pencils may come in handy but are not required.
\item \textbf{Scientific calculator:}  A calculator capable of performing trigonometric functions, logarithms, exponents, roots, etc. A graphing calculator is not required. 
\item \textbf{Laptop:} Laptops will be a necessity for many of the labs.  A limited number of laptops will be available for students who don't have their own. \\
\end{itemize}
 
\section*{Grading}

\subsection*{Lab Write-ups:}
\noindent I will clearly denote what I would like you to record in your write-ups for each lab (see below). Lab responses can be recorded in a bound physical notebook, but we strongly prefer digital \textit{submissions} for your lab write-ups. These don't have to be typed; feel free to hand-write your work in a bound notebook and upload (clear, legible) photos of the pages to CourseWorks.  All submissions will be due by midnight on the day of the lab. For students who don't utilize digital submissions, notebooks will be collected at the end of every lab session and returned at the beginning of the next lab. I will collect your notebooks at the end of each lab session and I will aim to return them to you, graded, by the next lab period. 

If you are working with a partner, each of you should keep your own records. The entire goal of the write-ups is to explain to the instructor \textit{what} you did during the lab, \textit{how} you did it, and \textit{why} you did it --- I am much more concerned with the coherence of your arguments/chain of reasoning than I am about the format. At the end of each session I will ask that you write a brief paragraph reflecting on the lab we completed that day: what you learned, anything that stood out to you or surprised you, and/or what part of the lab you found most interesting (and why). No write-up will be required for the final session, which will be dedicated to student presentations. Your lowest lab grade will be dropped when determining your final grade.

 
\subsubsection*{Lab write-up guidelines:}
 
\begin{itemize}
\item[--] Each lab's write-up should begin on a new page and have your name, your partner's name, lab title, and the date at the top.
\item[--] State specifically and in detail what your assumptions, methods, calculations, observations and conclusions are.
\item[--] Clearly mark (e.g. underline/highlight/put a box around) your final answer. However, note that the exact value is less important than the methods you used to obtain it.
\item[--] Always include units! Using them throughout your calculations will help you keep track of what you're doing and prevent careless mistakes. 
\item[--] Units should be appropriate for the object evaluated; e.g. don't measure the Sun in centimeters unless (instructed). The values of any constants must match your units!
\item[--] Plots should have both axes labelled with units, and a legend or other indication of what each symbol/line represents.
\item[--] Ensure that your handwriting is legible. If we cannot read it easily, we won't be able to grade it and you will not get any credit. \\
\end{itemize}
 
\subsection*{Participation:}
\noindent Participation is an essential part of this lab. You will often work in pairs or groups of three (although given the present COVID-19 situation, students will be given the option to work alone if they prefer to do so to reduce the risk of transmission). I ask that each week you try to work with someone new, so that by the end of the semester you will have the opportunity to work with all of your classmates. Your participation grade will be based on how well you work with your group, if you come to lab prepared and on time, the number of questions you ask, initiation of class discussions, and attempts at answering any questions that the instructors or other students pose. Be creative with how you can effectively participate in the class proceedings! Please do not use your phone during class (unless absolutely unavoidable).\\

\subsection*{Final Presentations:}
\noindent For the final session, each student will give a 10-minute presentation followed by a 5-minute discussion with the class. A list of topics related to astronomy and science in society will be rolled out; you are also welcome to submit your own suggested topics for instructor approval. The presentation will be worth the weight of two lab sessions.  \\ 

\bigskip

\noindent Your final grade will be determined as follows:
\begin{itemize}
\item 70\% Lab write-ups 
\item 15\% Participation 
\item 15\% Final presentation \\
\end{itemize}
 
 
\section*{Policies:}
 
\subsection*{Attendance}
 
It is important that you are present \textbf{on time}. Since your grade incorporates 9 out of 10 labs, you may either miss one class (without a valid excuse) or have your lowest lab grade dropped. We strongly recommend the latter. By department policy, more than two unexcused (non-medical, non-quarantine-related) absences will result in automatic failure of the course. We understand that extenuating circumstances may arise; in cases of family emergencies, serious illness, quarantine requirement, or religious holidays, please notify one of the instructors \textit{before} the missed lab and we will arrange a make-up lab. 
 
\subsection*{Accommodations}
Please speak with one of the instructors if this course can be better adapted to your needs without sacrificing the integrity of instruction. If you have an identified disability, we encourage you to register with the Office of Disability Services early in the semester to ensure access to any necessary resources; registration is confidential.
 
\subsection*{Academic honesty}
Do not falsify data. Give credit to others' work. Do not present text verbatim from other sources as if it
were your own; do not otherwise plagiarize. Please ask if you are unsure what's acceptable. Academic honesty is taken very seriously and will lead to penalties if not adhered to.
 
\subsection*{Mandatory reporting}
Instructors are required to report allegations of ``gender based misconduct, discrimination, or harassment" to Columbia's administration. While I am more than willing to listen and seek out resources (including confidential counselors) on your behalf, I cannot myself provide confidentiality.
 
\section*{Other astronomy related events at Columbia:}
Public lectures and observing sessions: http://outreach.astro.columbia.edu/

\end{document}
